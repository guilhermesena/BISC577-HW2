\documentclass[11pt]{article}

\usepackage{fullpage}
\usepackage{amsmath}
\usepackage{amssymb}
\usepackage{mathtools}
\usepackage{graphicx}
\usepackage{hyperref}
\setlength\parindent{0pt}

\title{BISC-577 Module 3 Assignment 2}
\author{Guilherme de Sena Brandine}

\begin{document}
\maketitle

\section{An open-source and distributed revision control project}
The entire project was done on GitHub and can be found at:\\
\\
\href{https://github.com/guilhermesena/BISC577-HW2}{https://github.com/guilhermesena/BISC577-HW2} \\
\\
The source code for each specific section of this report are on the root folder and called \textbf{question\_x.r} where $x \in \{ 4,5,6,7,8 \}$ and this report can be found in the \textbf{report} directory. All input files (fasta files or the ones generated by DNAshapeR) are located in the \textbf{in} folder except for the AH28451 CTCF binding site regions on question 6 due to its large file size. 


\section{High throughput binding assays}
\textbf{SELEX-Seq} (Systematic Evolution of Ligands by Exponential Enrichment, aka in vitro selection): It is a method for characterizing the complete repertoire of binding site preference for transcription factor complexes. It consists of constructing a DNA library on an array and introducing the desired TF to the array, which allows us to quantify which k-mers the protein prefers to bind to, the major drawback being that we are limited to small k-mers (10 to 14 max) as the number of possible k-mers grows exponentially with the k-mer size.\\
\\
\textbf{PBM} (Protein Binding Microarray): In terms of experimental design, PBM is the reverse of SELEX and measures the DNA affinity to transcription factors. This time proteins are the ones bound to an array and DNA probes are allowed to flow freely in the solution. If they bind to the array proteins, they can be later amplified by PCR and sequenced to quantify the k-mers that have binding specificity to the proteins in the array. The disadvantage of this method is that we can't do genome-wide studies of TF binding through this method as we need to have previous knowledge of what proteins to insert into the array. \\
\\
\textbf{ChIp-Seq}: Unlike the former experiments, ChIp-Seq (Chromatin Immunoprecipitation followed by Sequenicng) is done in vivo. The idea is to use formaldehyde to cross-link proteins to the DNA and subsequently use antibodies specific to the desired protein (ie, TFs, histones, enhancers, etc) to immunoprecipitate the DNA-protein complex. We can then sequence the DNA regions that are IP-ed with the protein to find enriched regions in the genome. \\

\section{Preparation of high throughput in vitro data analysis}

I used the r portal to download R-3.3.0 for unix and the bioconductor source within R to download DNAshapeR through the biocLite() module. I also use the built-in install.packages() module to download caret from the R could mirror.

\section{Build prediction models for in vitro data}

The source code for this problem can be found at file \textbf{question\_4.r}. I used caret to train a model through L2-regularization to predict the binding affinity of DNA sequences. I compared two cases: using only the 1-mer feature or the 1-mer+shape by using shape data from the DNA, and the makeSummary() function in the source code returns the maximum average R-squared for the simulation. These are the values I obtained for the 3 given input files: \\

\begin{center}
\begin{tabular}{ | l | c | c | r | }
\hline
   & Mad & Max & Myc \\
\hline
  1-mer & 0.775 & 0.785 & 0.778 \\
\hline  
  1-shape & 0.863 & 0.864 & 0.855 \\
\hline
\end{tabular}
\end{center}
\\
We notice qualitatively that adding the shape feature increases the model performance. 

\section{High throughput in vitro data analysis}
\section{Preparation of high throughput in vivo data analysis}
\section{High throughput in vivo data analysis}
\section{Build prediction models for in vivo data analysis}

\end{document}
